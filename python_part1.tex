% Template for beamer class with section and label examples
\documentclass{beamer}

\usepackage{adjustbox}
\usepackage{graphicx}
\usepackage{xcolor}
\usepackage{ragged2e}
\usepackage{tikz}                 	% transparency
\usepackage{upquote} 		% Upright quotes for verbatim code

\usepackage[T1]{fontenc}
% Nicer default font (+ math font) than Computer Modern for most use cases
\usepackage{mathpazo}

    % We will generate all images so they have a width \maxwidth. This means
    % that they will get their normal width if they fit onto the page, but
    % are scaled down if they would overflow the margins.
    \makeatletter
    \def\maxwidth{\ifdim\Gin@nat@width>\linewidth\linewidth
    \else\Gin@nat@width\fi}
    \makeatother
    \let\Oldincludegraphics\includegraphics
    % Set max figure width to be 80% of text width, for now hardcoded.
    \renewcommand{\includegraphics}[1]{\Oldincludegraphics[width=.8\maxwidth]{#1}}
    % Ensure that by default, figures have no caption (until we provide a
    % proper Figure object with a Caption API and a way to capture that
    % in the conversion process - todo).
    \usepackage{caption}
    \DeclareCaptionLabelFormat{nolabel}{}
    \captionsetup{labelformat=nolabel}

    \usepackage{adjustbox} % Used to constrain images to a maximum size 
    \usepackage{xcolor} % Allow colors to be defined
    \usepackage{enumerate} % Needed for markdown enumerations to work
    \usepackage{geometry} % Used to adjust the document margins
    \usepackage{amsmath} % Equations
    \usepackage{amssymb} % Equations
    \usepackage{textcomp} % defines textquotesingle
    % Hack from http://tex.stackexchange.com/a/47451/13684:
    \AtBeginDocument{%
        \def\PYZsq{\textquotesingle}% Upright quotes in Pygmentized code
    }
    \usepackage{upquote} % Upright quotes for verbatim code
    \usepackage[mathletters]{ucs} % Extended unicode (utf-8) support
    \usepackage[utf8x]{inputenc} % Allow utf-8 characters in the tex document
    \usepackage{fancyvrb} % verbatim replacement that allows latex
    \usepackage{grffile} % extends the file name processing of package graphics 
                         % to support a larger range 
    % The hyperref package gives us a pdf with properly built
    % internal navigation ('pdf bookmarks' for the table of contents,
    % internal cross-reference links, web links for URLs, etc.)
    \usepackage{hyperref}
    \usepackage{longtable} % longtable support required by pandoc >1.10
    \usepackage{booktabs}  % table support for pandoc > 1.12.2
    %\usepackage[inline]{enumitem} % IRkernel/repr support (it uses the enumerate* environment)
    \usepackage[normalem]{ulem} % ulem is needed to support strikethroughs (\sout)
                                % normalem makes italics be italics, not underlines

    % Colors for the hyperref package
    \definecolor{urlcolor}{rgb}{0,.145,.698}
    \definecolor{linkcolor}{rgb}{.71,0.21,0.01}
    \definecolor{citecolor}{rgb}{.12,.54,.11}

    % ANSI colors
    \definecolor{ansi-black}{HTML}{3E424D}
    \definecolor{ansi-black-intense}{HTML}{282C36}
    \definecolor{ansi-red}{HTML}{E75C58}
    \definecolor{ansi-red-intense}{HTML}{B22B31}
    \definecolor{ansi-green}{HTML}{00A250}
    \definecolor{ansi-green-intense}{HTML}{007427}
    \definecolor{ansi-yellow}{HTML}{DDB62B}
    \definecolor{ansi-yellow-intense}{HTML}{B27D12}
    \definecolor{ansi-blue}{HTML}{208FFB}
    \definecolor{ansi-blue-intense}{HTML}{0065CA}
    \definecolor{ansi-magenta}{HTML}{D160C4}
    \definecolor{ansi-magenta-intense}{HTML}{A03196}
    \definecolor{ansi-cyan}{HTML}{60C6C8}
    \definecolor{ansi-cyan-intense}{HTML}{258F8F}
    \definecolor{ansi-white}{HTML}{C5C1B4}
    \definecolor{ansi-white-intense}{HTML}{A1A6B2}

    % commands and environments needed by pandoc snippets
    % extracted from the output of `pandoc -s`
    \providecommand{\tightlist}{%
      \setlength{\itemsep}{0pt}\setlength{\parskip}{0pt}}
    \DefineVerbatimEnvironment{Highlighting}{Verbatim}{commandchars=\\\{\}}
    % Add ',fontsize=\small' for more characters per line ,
    \newenvironment{Shaded}{}{}
    \newcommand{\KeywordTok}[1]{\textcolor[rgb]{0.00,0.44,0.13}{\textbf{{#1}}}}
    \newcommand{\DataTypeTok}[1]{\textcolor[rgb]{0.56,0.13,0.00}{{#1}}}
    \newcommand{\DecValTok}[1]{\textcolor[rgb]{0.25,0.63,0.44}{{#1}}}
    \newcommand{\BaseNTok}[1]{\textcolor[rgb]{0.25,0.63,0.44}{{#1}}}
    \newcommand{\FloatTok}[1]{\textcolor[rgb]{0.25,0.63,0.44}{{#1}}}
    \newcommand{\CharTok}[1]{\textcolor[rgb]{0.25,0.44,0.63}{{#1}}}
    \newcommand{\StringTok}[1]{\textcolor[rgb]{0.25,0.44,0.63}{{#1}}}
    \newcommand{\CommentTok}[1]{\textcolor[rgb]{0.38,0.63,0.69}{\textit{{#1}}}}
    \newcommand{\OtherTok}[1]{\textcolor[rgb]{0.00,0.44,0.13}{{#1}}}
    \newcommand{\AlertTok}[1]{\textcolor[rgb]{1.00,0.00,0.00}{\textbf{{#1}}}}
    \newcommand{\FunctionTok}[1]{\textcolor[rgb]{0.02,0.16,0.49}{{#1}}}
    \newcommand{\RegionMarkerTok}[1]{{#1}}
    \newcommand{\ErrorTok}[1]{\textcolor[rgb]{1.00,0.00,0.00}{\textbf{{#1}}}}
    \newcommand{\NormalTok}[1]{{#1}}
    
    % Additional commands for more recent versions of Pandoc
    \newcommand{\ConstantTok}[1]{\textcolor[rgb]{0.53,0.00,0.00}{{#1}}}
    \newcommand{\SpecialCharTok}[1]{\textcolor[rgb]{0.25,0.44,0.63}{{#1}}}
    \newcommand{\VerbatimStringTok}[1]{\textcolor[rgb]{0.25,0.44,0.63}{{#1}}}
    \newcommand{\SpecialStringTok}[1]{\textcolor[rgb]{0.73,0.40,0.53}{{#1}}}
    \newcommand{\ImportTok}[1]{{#1}}
    \newcommand{\DocumentationTok}[1]{\textcolor[rgb]{0.73,0.13,0.13}{\textit{{#1}}}}
    \newcommand{\AnnotationTok}[1]{\textcolor[rgb]{0.38,0.63,0.69}{\textbf{\textit{{#1}}}}}
    \newcommand{\CommentVarTok}[1]{\textcolor[rgb]{0.38,0.63,0.69}{\textbf{\textit{{#1}}}}}
    \newcommand{\VariableTok}[1]{\textcolor[rgb]{0.10,0.09,0.49}{{#1}}}
    \newcommand{\ControlFlowTok}[1]{\textcolor[rgb]{0.00,0.44,0.13}{\textbf{{#1}}}}
    \newcommand{\OperatorTok}[1]{\textcolor[rgb]{0.40,0.40,0.40}{{#1}}}
    \newcommand{\BuiltInTok}[1]{{#1}}
    \newcommand{\ExtensionTok}[1]{{#1}}
    \newcommand{\PreprocessorTok}[1]{\textcolor[rgb]{0.74,0.48,0.00}{{#1}}}
    \newcommand{\AttributeTok}[1]{\textcolor[rgb]{0.49,0.56,0.16}{{#1}}}
    \newcommand{\InformationTok}[1]{\textcolor[rgb]{0.38,0.63,0.69}{\textbf{\textit{{#1}}}}}
    \newcommand{\WarningTok}[1]{\textcolor[rgb]{0.38,0.63,0.69}{\textbf{\textit{{#1}}}}}
    
    
    % Define a nice break command that doesn't care if a line doesn't already
    % exist.
    \def\br{\hspace*{\fill} \\* }
    % Math Jax compatability definitions
    \def\gt{>}
    \def\lt{<}
    % Document parameters
    \title{1 - Jupyter}

    % Pygments definitions
    
\makeatletter
\def\PY@reset{\let\PY@it=\relax \let\PY@bf=\relax%
    \let\PY@ul=\relax \let\PY@tc=\relax%
    \let\PY@bc=\relax \let\PY@ff=\relax}
\def\PY@tok#1{\csname PY@tok@#1\endcsname}
\def\PY@toks#1+{\ifx\relax#1\empty\else%
    \PY@tok{#1}\expandafter\PY@toks\fi}
\def\PY@do#1{\PY@bc{\PY@tc{\PY@ul{%
    \PY@it{\PY@bf{\PY@ff{#1}}}}}}}
\def\PY#1#2{\PY@reset\PY@toks#1+\relax+\PY@do{#2}}

\expandafter\def\csname PY@tok@gd\endcsname{\def\PY@tc##1{\textcolor[rgb]{0.63,0.00,0.00}{##1}}}
\expandafter\def\csname PY@tok@gu\endcsname{\let\PY@bf=\textbf\def\PY@tc##1{\textcolor[rgb]{0.50,0.00,0.50}{##1}}}
\expandafter\def\csname PY@tok@gt\endcsname{\def\PY@tc##1{\textcolor[rgb]{0.00,0.27,0.87}{##1}}}
\expandafter\def\csname PY@tok@gs\endcsname{\let\PY@bf=\textbf}
\expandafter\def\csname PY@tok@gr\endcsname{\def\PY@tc##1{\textcolor[rgb]{1.00,0.00,0.00}{##1}}}
\expandafter\def\csname PY@tok@cm\endcsname{\let\PY@it=\textit\def\PY@tc##1{\textcolor[rgb]{0.25,0.50,0.50}{##1}}}
\expandafter\def\csname PY@tok@vg\endcsname{\def\PY@tc##1{\textcolor[rgb]{0.10,0.09,0.49}{##1}}}
\expandafter\def\csname PY@tok@vi\endcsname{\def\PY@tc##1{\textcolor[rgb]{0.10,0.09,0.49}{##1}}}
\expandafter\def\csname PY@tok@mh\endcsname{\def\PY@tc##1{\textcolor[rgb]{0.40,0.40,0.40}{##1}}}
\expandafter\def\csname PY@tok@cs\endcsname{\let\PY@it=\textit\def\PY@tc##1{\textcolor[rgb]{0.25,0.50,0.50}{##1}}}
\expandafter\def\csname PY@tok@ge\endcsname{\let\PY@it=\textit}
\expandafter\def\csname PY@tok@vc\endcsname{\def\PY@tc##1{\textcolor[rgb]{0.10,0.09,0.49}{##1}}}
\expandafter\def\csname PY@tok@il\endcsname{\def\PY@tc##1{\textcolor[rgb]{0.40,0.40,0.40}{##1}}}
\expandafter\def\csname PY@tok@go\endcsname{\def\PY@tc##1{\textcolor[rgb]{0.53,0.53,0.53}{##1}}}
\expandafter\def\csname PY@tok@cp\endcsname{\def\PY@tc##1{\textcolor[rgb]{0.74,0.48,0.00}{##1}}}
\expandafter\def\csname PY@tok@gi\endcsname{\def\PY@tc##1{\textcolor[rgb]{0.00,0.63,0.00}{##1}}}
\expandafter\def\csname PY@tok@gh\endcsname{\let\PY@bf=\textbf\def\PY@tc##1{\textcolor[rgb]{0.00,0.00,0.50}{##1}}}
\expandafter\def\csname PY@tok@ni\endcsname{\let\PY@bf=\textbf\def\PY@tc##1{\textcolor[rgb]{0.60,0.60,0.60}{##1}}}
\expandafter\def\csname PY@tok@nl\endcsname{\def\PY@tc##1{\textcolor[rgb]{0.63,0.63,0.00}{##1}}}
\expandafter\def\csname PY@tok@nn\endcsname{\let\PY@bf=\textbf\def\PY@tc##1{\textcolor[rgb]{0.00,0.00,1.00}{##1}}}
\expandafter\def\csname PY@tok@no\endcsname{\def\PY@tc##1{\textcolor[rgb]{0.53,0.00,0.00}{##1}}}
\expandafter\def\csname PY@tok@na\endcsname{\def\PY@tc##1{\textcolor[rgb]{0.49,0.56,0.16}{##1}}}
\expandafter\def\csname PY@tok@nb\endcsname{\def\PY@tc##1{\textcolor[rgb]{0.00,0.50,0.00}{##1}}}
\expandafter\def\csname PY@tok@nc\endcsname{\let\PY@bf=\textbf\def\PY@tc##1{\textcolor[rgb]{0.00,0.00,1.00}{##1}}}
\expandafter\def\csname PY@tok@nd\endcsname{\def\PY@tc##1{\textcolor[rgb]{0.67,0.13,1.00}{##1}}}
\expandafter\def\csname PY@tok@ne\endcsname{\let\PY@bf=\textbf\def\PY@tc##1{\textcolor[rgb]{0.82,0.25,0.23}{##1}}}
\expandafter\def\csname PY@tok@nf\endcsname{\def\PY@tc##1{\textcolor[rgb]{0.00,0.00,1.00}{##1}}}
\expandafter\def\csname PY@tok@si\endcsname{\let\PY@bf=\textbf\def\PY@tc##1{\textcolor[rgb]{0.73,0.40,0.53}{##1}}}
\expandafter\def\csname PY@tok@s2\endcsname{\def\PY@tc##1{\textcolor[rgb]{0.73,0.13,0.13}{##1}}}
\expandafter\def\csname PY@tok@nt\endcsname{\let\PY@bf=\textbf\def\PY@tc##1{\textcolor[rgb]{0.00,0.50,0.00}{##1}}}
\expandafter\def\csname PY@tok@nv\endcsname{\def\PY@tc##1{\textcolor[rgb]{0.10,0.09,0.49}{##1}}}
\expandafter\def\csname PY@tok@s1\endcsname{\def\PY@tc##1{\textcolor[rgb]{0.73,0.13,0.13}{##1}}}
\expandafter\def\csname PY@tok@ch\endcsname{\let\PY@it=\textit\def\PY@tc##1{\textcolor[rgb]{0.25,0.50,0.50}{##1}}}
\expandafter\def\csname PY@tok@m\endcsname{\def\PY@tc##1{\textcolor[rgb]{0.40,0.40,0.40}{##1}}}
\expandafter\def\csname PY@tok@gp\endcsname{\let\PY@bf=\textbf\def\PY@tc##1{\textcolor[rgb]{0.00,0.00,0.50}{##1}}}
\expandafter\def\csname PY@tok@sh\endcsname{\def\PY@tc##1{\textcolor[rgb]{0.73,0.13,0.13}{##1}}}
\expandafter\def\csname PY@tok@ow\endcsname{\let\PY@bf=\textbf\def\PY@tc##1{\textcolor[rgb]{0.67,0.13,1.00}{##1}}}
\expandafter\def\csname PY@tok@sx\endcsname{\def\PY@tc##1{\textcolor[rgb]{0.00,0.50,0.00}{##1}}}
\expandafter\def\csname PY@tok@bp\endcsname{\def\PY@tc##1{\textcolor[rgb]{0.00,0.50,0.00}{##1}}}
\expandafter\def\csname PY@tok@c1\endcsname{\let\PY@it=\textit\def\PY@tc##1{\textcolor[rgb]{0.25,0.50,0.50}{##1}}}
\expandafter\def\csname PY@tok@o\endcsname{\def\PY@tc##1{\textcolor[rgb]{0.40,0.40,0.40}{##1}}}
\expandafter\def\csname PY@tok@kc\endcsname{\let\PY@bf=\textbf\def\PY@tc##1{\textcolor[rgb]{0.00,0.50,0.00}{##1}}}
\expandafter\def\csname PY@tok@c\endcsname{\let\PY@it=\textit\def\PY@tc##1{\textcolor[rgb]{0.25,0.50,0.50}{##1}}}
\expandafter\def\csname PY@tok@mf\endcsname{\def\PY@tc##1{\textcolor[rgb]{0.40,0.40,0.40}{##1}}}
\expandafter\def\csname PY@tok@err\endcsname{\def\PY@bc##1{\setlength{\fboxsep}{0pt}\fcolorbox[rgb]{1.00,0.00,0.00}{1,1,1}{\strut ##1}}}
\expandafter\def\csname PY@tok@mb\endcsname{\def\PY@tc##1{\textcolor[rgb]{0.40,0.40,0.40}{##1}}}
\expandafter\def\csname PY@tok@ss\endcsname{\def\PY@tc##1{\textcolor[rgb]{0.10,0.09,0.49}{##1}}}
\expandafter\def\csname PY@tok@sr\endcsname{\def\PY@tc##1{\textcolor[rgb]{0.73,0.40,0.53}{##1}}}
\expandafter\def\csname PY@tok@mo\endcsname{\def\PY@tc##1{\textcolor[rgb]{0.40,0.40,0.40}{##1}}}
\expandafter\def\csname PY@tok@kd\endcsname{\let\PY@bf=\textbf\def\PY@tc##1{\textcolor[rgb]{0.00,0.50,0.00}{##1}}}
\expandafter\def\csname PY@tok@mi\endcsname{\def\PY@tc##1{\textcolor[rgb]{0.40,0.40,0.40}{##1}}}
\expandafter\def\csname PY@tok@kn\endcsname{\let\PY@bf=\textbf\def\PY@tc##1{\textcolor[rgb]{0.00,0.50,0.00}{##1}}}
\expandafter\def\csname PY@tok@cpf\endcsname{\let\PY@it=\textit\def\PY@tc##1{\textcolor[rgb]{0.25,0.50,0.50}{##1}}}
\expandafter\def\csname PY@tok@kr\endcsname{\let\PY@bf=\textbf\def\PY@tc##1{\textcolor[rgb]{0.00,0.50,0.00}{##1}}}
\expandafter\def\csname PY@tok@s\endcsname{\def\PY@tc##1{\textcolor[rgb]{0.73,0.13,0.13}{##1}}}
\expandafter\def\csname PY@tok@kp\endcsname{\def\PY@tc##1{\textcolor[rgb]{0.00,0.50,0.00}{##1}}}
\expandafter\def\csname PY@tok@w\endcsname{\def\PY@tc##1{\textcolor[rgb]{0.73,0.73,0.73}{##1}}}
\expandafter\def\csname PY@tok@kt\endcsname{\def\PY@tc##1{\textcolor[rgb]{0.69,0.00,0.25}{##1}}}
\expandafter\def\csname PY@tok@sc\endcsname{\def\PY@tc##1{\textcolor[rgb]{0.73,0.13,0.13}{##1}}}
\expandafter\def\csname PY@tok@sb\endcsname{\def\PY@tc##1{\textcolor[rgb]{0.73,0.13,0.13}{##1}}}
\expandafter\def\csname PY@tok@k\endcsname{\let\PY@bf=\textbf\def\PY@tc##1{\textcolor[rgb]{0.00,0.50,0.00}{##1}}}
\expandafter\def\csname PY@tok@se\endcsname{\let\PY@bf=\textbf\def\PY@tc##1{\textcolor[rgb]{0.73,0.40,0.13}{##1}}}
\expandafter\def\csname PY@tok@sd\endcsname{\let\PY@it=\textit\def\PY@tc##1{\textcolor[rgb]{0.73,0.13,0.13}{##1}}}

\def\PYZbs{\char`\\}
\def\PYZus{\char`\_}
\def\PYZob{\char`\{}
\def\PYZcb{\char`\}}
\def\PYZca{\char`\^}
\def\PYZam{\char`\&}
\def\PYZlt{\char`\<}
\def\PYZgt{\char`\>}
\def\PYZsh{\char`\#}
\def\PYZpc{\char`\%}
\def\PYZdl{\char`\$}
\def\PYZhy{\char`\-}
\def\PYZsq{\char`\'}
\def\PYZdq{\char`\"}
\def\PYZti{\char`\~}
% for compatibility with earlier versions
\def\PYZat{@}
\def\PYZlb{[}
\def\PYZrb{]}
\makeatother


    % Exact colors from NB
    \definecolor{incolor}{rgb}{0.0, 0.0, 0.5}
    \definecolor{outcolor}{rgb}{0.545, 0.0, 0.0}

%beamer theme and options
\usetheme{Copenhagen}

\setbeamertemplate{itemize items}[default]
\setbeamertemplate{enumerate items}[default]

\AtBeginSection[]{
  \begin{frame}
  \vfill
  \centering
  \begin{beamercolorbox}[sep=8pt,center,shadow=true,rounded=true]{title}
    \usebeamerfont{title}\insertsectionhead\par%
  \end{beamercolorbox}
  \vfill
  \end{frame}
}

%%%%%%%%%%%%%%%%%%%%%%%%%%%
%          logo                                                                           %
%%%%%%%%%%%%%%%%%%%%%%%%%%%
% TODO

% the document
\begin{document}
\title{Python Workshop}   
\author{Jean-Mathieu Vermosen} 
\date{\today} 

\frame{\titlepage}

\begin{frame}[allowframebreaks]
\frametitle{Table of contents}
\tableofcontents[part=1]
\framebreak
\tableofcontents[part=2]
\framebreak
\tableofcontents[part=3]
\end{frame}

\part{1}
\section{A short introduction to Jupyter Notebook} 
\subsection{The Jupyter Project}
\frame{
\frametitle{The Jupyter Project} 
\justifying
Jupyter is a spin-off of the iPython project. It could be defined as
\emph{a command shell for interactive computing in multiple programming
languages, originally developed for the Python programming language,
that offers introspection, rich media, shell syntax, tab completion, and
history.} (wikipedia).
}
\subsection{Jupyter Notebook Modes}
\frame{ 
\frametitle{Jupyter Notebook Modes} 
\justifying
\begin{itemize}
\item A Jupyter notebook is made of \emph{cells}. Each cell can be displayed in two different modes: \textbf{edition} and \textbf{command}.
\item You can figure out at any time which mode is activated by looking at the
color of the selection (green for edition, blue for command mode).
\item To toggle edit mode, press \textbf{Enter}. To toggle command mode, press
\textbf{Esc} (or ctrl + m)
\end{itemize} 
}

\frame{ 
A cell displayed in the edition mode...
%\begin{center}
%\adjustimage{max size={0.9\linewidth}{0.9\paperheight}}{./images/pict_2.png}
%\end{center}
}

\frame{
The same cell when command mode is turned on...
%\begin{center}
%\adjustimage{max size={0.9\linewidth}{0.9\paperheight}}{./images/pict_1.png}
%\end{center}
}

\subsubsection{The Command Mode}\label{the-edit-mode}
\frame{ 
\frametitle{The Command Mode} 
\justifying
The command mode is used to edit how cells are ordered and to control there behavior.
Basic edition commands are:
\begin{itemize}
\item add/delete cells
\begin{itemize}
\item \textbf{a} add a cell above
\item \textbf{b} add a cell below
\item \textbf{d,d} add a cell below
\end{itemize}
\item copy/paste
\begin{itemize}
\item \textbf{c} copy
\item \textbf{x} cut
\item \textbf{v} past below selection
\item \textbf{shift + v} past above selection
\end{itemize}
\end{itemize}
}

\frame{ 
\frametitle{The Command Mode} 
\justifying
Basic notebooks are composed of \textbf{cells} that may be of \textbf{two
kinds}: \emph{markdown} or \emph{code}.
\begin{itemize}
\item \textbf{Markdown cells} embed
rich text features with a simplified syntax close to HTML (as opposed to
WYSIWYG \footnote{What you see is what you get} approaches). 
\item \textbf{Code cells} contain code that can be run from the current kernel (Python in our case)
\end{itemize}
Cells have to be specifically switched from one mode into another by the user
\begin{itemize}
\item \textbf{y} turns a markdown cell into a code cell
\item \textbf{m} turns a code cell in a markdown cell
\end{itemize}
}

\subsubsection{The Edition Mode}\label{the-edition-mode}

\frame{ 
\frametitle{The Edition Mode} 
\justifying
The edition mode is used to edit the \textbf{cells content}. The cells are
composed of unformatted text blocks from possibly several languages. The
usual text editor shortcut applies (cut, copy, paste, tab). Once a cell
has been edited, it can be "Executed" from the interface or using the key
combinaison \textbf{Shift + Enter}
}

\subsection{Notebook Cell Structure}

\frame{ 
\frametitle{Notebook Cell Structure} 
\justifying
Jupyter cells behavior and content depends of their nature. 
\begin{itemize}
\item The \textbf{Markdown cells} embed rich format features. 
\item The \textbf{code cells} are set to interact with a specific kernel by executing code commands.
\end{itemize}
}

\subsubsection{Markdown Cells}
\frame{ 
\frametitle{Markdown Cells} 
\justifying
The cells can contain many rich features such as:
\begin{itemize}
\item Titles, \emph{formatted} \textbf{text}
\item
  Lists
 \item Hyperlinks: \textcolor{blue}{\href{https://en.wikipedia.org/wiki/IPython}{wikipedia}}
\item Alerts
\begin{exampleblock}{Well Done !}
\end{exampleblock}
\item images, videos, gifs, blockquotes, etc.
\end{itemize}
"Running" a markdown cell will only gathered the objects contained and apply the formatting, so there is no risk to trigger any computation or change.
}


\subsubsection{Latex Rendering}

\frame{ 
\frametitle{Latex Rendering} 
\justifying
Mathematical notations are supported through
\href{https://www.mathjax.org/}{MathJax} which is a browser-specific
implementation of \LaTeX. Being a subproject of the latter, not all of the
packages are supported, but it will provide a pretty good coverage of
the main rendering classes. Equations can be inserted in the body of the
document using $\$$. For example,
$\oint_C \frac {f \left({z}\right)} {z - z_0} \ \mathrm d z = 2 i \pi f \left({z_0}\right)$
or create a block using $\$\$$
\begin{equation}
  \int_0^\infty \frac{x^3}{e^x-1}\,dx = \frac{\pi^4}{15}
\end{equation}
the main \LaTeX modes are supporter such as tabular, eqnarray, etc.
}

\subsubsection{Code Cells}
\frame{ 
\frametitle{Code Cells} 
\justifying
Code cells contain... code ! For those cells, pressing Shift + Enter triggers the execution in the
runtime and the output will display below the cell.
\\~\\ Originally, iPython only supported a
Python runtime. The original Jupyter project widen the scope to 2
additional interpreted languages: Julia and R
(\textbf{JuPy}te\textbf{R}).The current tool supports 40 languages
including compiled one such as C++ and Fortran.
}

\begin{frame}[fragile]
The following code cell
\begin{Verbatim}[commandchars=\\\{\}]
{\color{incolor}In [{\color{incolor}5}]:} \PY{c+c1}{\PYZsh{} a first example in Python}
        \PY{k+kn}{from} \PY{n+nn}{matplotlib} \PY{k+kn}{import} \PY{n}{pyplot} \PY{k}{as} \PY{n}{plt}
        \PY{k+kn}{import} \PY{n+nn}{numpy} \PY{k+kn}{as} \PY{n+nn}{np}
        
        \PY{n}{X} \PY{o}{=} \PY{n}{np}\PY{o}{.}\PY{n}{linspace}\PY{p}{(}\PY{o}{\PYZhy{}}\PY{l+m+mi}{3}\PY{p}{,} \PY{l+m+mi}{3}\PY{p}{,} \PY{l+m+mi}{256}\PY{p}{)}
        \PY{n}{S} \PY{o}{=} \PY{n}{np}\PY{o}{.}\PY{n}{sin}\PY{p}{(}\PY{n}{X}\PY{p}{)}
        
        \PY{n}{plt}\PY{o}{.}\PY{n}{plot}\PY{p}{(}\PY{n}{X}\PY{p}{,} \PY{n}{S}\PY{p}{)}
        \PY{n}{plt}\PY{o}{.}\PY{n}{title}\PY{p}{(}\PY{l+s+s2}{\PYZdq{}}\PY{l+s+s2}{y=sin(x)}\PY{l+s+s2}{\PYZdq{}}\PY{p}{)}
        \PY{n}{plt}\PY{o}{.}\PY{n}{show}\PY{p}{(}\PY{p}{)}
\end{Verbatim}
\end{frame}

\subsubsection{Code Cells}
\frame{ 
Hitting \textbf{Shift+enter} will trigger the execution of the background runtime:
%\begin{center}
%\adjustimage{max size={0.9\linewidth}{0.9\paperheight}}{./images/1 - Jupyter_6_0.png}
%\end{center}
{ \hspace*{\fill} \\}
}

\subsection{Jupyter Magic Commands}

\frame{ 
\frametitle{Jupyter Magic Commands} 
\justifying
As part of the Jupyter framework, there are useful commands that can be
run in code cells to handle specific actions along the code.
\\~\\
Magic commands \textbf{have to be inserted in code cells}. They are not
specific to any language and won't be sent to the iPython kernel. Hence,
they \textbf{won't interfere with the program executed}.
}

\frame{ 
\frametitle{General Purpose Functions} 
\textbf{\%quickref}, \textbf{\%lsmagic}, \textbf{\%magic} Gives help on the magic commands
}

\frame{ 
\frametitle{Environment commands} 
\begin{itemize}
\item
\textbf{\%reset {[}-f{]}} Resets the notebook variables. The -f option suppress the prompt
\item
\textbf{\%cd} changes the current directory
\item
\textbf{\%env} add an environment variable
\textbf{\%notebook} Save the code contained in the notebook as a standalone file
\end{itemize}
}

\frame{ 
\frametitle{Debugging} 
\begin{itemize}
\item
\textbf{\%time} Print the execution time of a given cell
\item
\textbf{\%debug} Launch the debugger
\item
\textbf{\%pdb} Control the way the interactive debugger can be called.
  For instance \textbf{\%pdb on} will trigger the debugger as soon as an exception
  has been thrown.
\end{itemize}
}

\part{2}
\section{ The Python Programming Language}
\subsection{A few words about Python}
\frame{ 
\justifying
\frametitle{A foreword on Python} 
Python is a high-level programming language used for general-purpose
programming, originally released by Guido van Rossum in 1991. An
interpreted language, Python has a design philosophy which emphasizes
\textbf{code readability}, and a syntax which allows programmers to
express concepts in \textbf{fewer lines of code} than possible in
languages.
\\~\\
Python has been widely adopted in the scientific community and the
financial industry. Hence, a large number of open-source libraries can
handle a wide variety of \textbf{statistical} and \textbf{numerical}
computation. It also offers a large choice of tools to bind existing
libraries.
}

\frame{
\justifying
Python comes in \textbf{two major versions}: 2.7 and 3.x (currently
3.6). The later, released in December 2008, introduced breaking changes
which required to rewrite most of the libraries. Hence, two versions
have been maintained and major libraries have been slowly ported in the
new language variant. 
\\~\\
The 2.7 branch is still \textbf{actively maintained} since the major language developements have been implemented
in both branches and potential bugs will be fixed until 2020. For these
reasons, Python 2.7 remains the version which is still mainly used in a
professional setup.
}
 
\subsection{The language syntax}
\subsubsection{Variables and assignement}
\frame{ 
\justifying
\frametitle{Variables and assignement} 
Variables in python are declared using the \textbf{assignement operator
=} (like in many programming languages). However, \textbf{variables are
never explicitely typed} and are always deduced at runtime from the
current context. Python variables do have a type which can be retrieved
using the \emph{type()} operator.
}

\begin{frame}[fragile]
\justifying
\begin{Verbatim}[commandchars=\\\{\}]
\PY{c+c1}{\PYZsh{} variable declaration}
\PY{n}{i} \PY{o}{=} \PY{l+m+mi}{5}
\PY{n}{j} \PY{o}{=} \PY{l+m+mf}{5.0}
\PY{n}{k} \PY{o}{=} \PY{l+s+s1}{\PYZsq{}}\PY{l+s+s1}{5}\PY{l+s+s1}{\PYZsq{}}
        
\PY{k}{print}\PY{p}{(}\PY{n+nb}{type}\PY{p}{(}\PY{n}{i}\PY{p}{)}\PY{p}{,} \PY{n+nb}{type}\PY{p}{(}\PY{n}{j}\PY{p}{)}\PY{p}{,} \PY{n+nb}{type}\PY{p}{(}\PY{n}{k}\PY{p}{)}\PY{p}{)}
\end{Verbatim}
output:
\begin{Verbatim}[commandchars=\\\{\}]
(<type 'int'>, <type 'float'>, <type 'str'>)
\end{Verbatim}
\end{frame}

\begin{frame}[fragile]
\justifying
Combining variables of different types may be a source of confusion:
when permitted, Python will try to \textbf{cast} one of the variable to make an
operation valid. This is not always possible. The following code
    \begin{Verbatim}[commandchars=\\\{\}]
\PY{c+c1}{\PYZsh{} this works}
\PY{n}{l} \PY{o}{=} \PY{n}{i} \PY{o}{+} \PY{n}{j} 
\PY{k}{print}\PY{p}{(}\PY{n}{l}\PY{p}{,} \PY{n+nb}{type}\PY{p}{(}\PY{n}{l}\PY{p}{)}\PY{p}{)}
        
\PY{c+c1}{\PYZsh{} this will create a runtime error}
\PY{n}{m} \PY{o}{=} \PY{n}{l} \PY{o}{+} \PY{n}{k}
\end{Verbatim}
\end{frame}

\begin{frame}[fragile]
\justifying
output: 
\begin{Verbatim}[commandchars=\\\{\}]
(10.0, <type 'float'>)

------------------------------------------------------
TypeErro     Traceback (most recent call last)

<ipython-input-8-335ccc39b24b> in <module>()
4 
5 \# this will create a runtime error
----> 6 m = l + k
TypeError: unsupported operand type(s) for +: 
'float' and 'str'
    \end{Verbatim}
\end{frame}

\begin{frame}[fragile]
\justifying
 In some case, an explicit cast might be needed to get the desired
result:

    \begin{Verbatim}[commandchars=\\\{\}]
\PY{c+c1}{\PYZsh{} this will work}
\PY{n}{right} \PY{o}{=} \PY{n}{l} \PY{o}{+} \PY{n+nb}{int}\PY{p}{(}\PY{n}{k}\PY{p}{)}
\PY{k}{print}\PY{p}{(}\PY{n}{right}\PY{p}{)}
        
\PY{c+c1}{\PYZsh{} beware of confusions !}
\PY{n}{wrong} \PY{o}{=} \PY{n}{k} \PY{o}{+} \PY{n+nb}{str}\PY{p}{(}\PY{n}{l}\PY{p}{)}
\PY{k}{print}\PY{p}{(}\PY{n}{wrong}\PY{p}{)}
\end{Verbatim}
output: 
    \begin{Verbatim}[commandchars=\\\{\}]
15.0
510.0
    \end{Verbatim}
\end{frame}

\subsubsection{Statements}
\begin{frame}[fragile]
\frametitle{Statements} 
\justifying
Python statements can be terminated in two ways. They are either
\textbf{carriage return} terminated, e.g. the statement ends at the end
of the line, but they can also be terminated by semicolon (;), which
allows multiple statement on the same line. The second syntax is rarely
used. 
\\~\\
The sharp (\#) sign allows one to insert \emph{comments}. Anytime
such a sign is hit, what follows one the same line will be totally
ignored.
\end{frame}

\begin{frame}[fragile]
\justifying
    \begin{Verbatim}[commandchars=\\\{\}]
\PY{c+c1}{\PYZsh{} This is a valid statement}
\PY{n}{a} \PY{o}{=} \PY{l+m+mi}{1}
         
\PY{n}{b} \PY{o}{=} \PY{l+m+mi}{2}      \PY{c+c1}{\PYZsh{} this is another valid statement}
         
\PY{c+c1}{\PYZsh{} this one will not be executed \PYZhy{}\PYZgt{} c = a + b}
         
\PY{c+c1}{\PYZsh{} we may also use semicolons to write}
\PY{c+c1}{\PYZsh{} a more compact code, but you should avoid it}
\PY{n}{c} \PY{o}{=} \PY{n}{a} \PY{o}{+} \PY{n}{b}\PY{p}{;} \PY{n}{d} \PY{o}{=} \PY{n}{c} \PY{o}{+} \PY{l+m+mi}{3}
\end{Verbatim}
\end{frame}

\subsubsection{Block Intendation}\label{block-intendation}

\begin{frame}[fragile]
\frametitle{Block Intendation} 
\justifying
Python use \textbf{whitespace indentation} to delimit blocks, rather
than curly braces like in C or keywords as in VBA.
\begin{itemize}
\item An \textbf{increase} in
indentation comes after certain statements
\item a \textbf{decrease} in
indentation signifies the \emph{end of the current block}
\end{itemize}

One of the simpliest block structure is a \textbf{try - except block}
which can be used to \emph{resume a program after an error occurs}.
\end{frame}

\begin{frame}[fragile]
\justifying
    \begin{Verbatim}[commandchars=\\\{\}]
\PY{c+c1}{\PYZsh{} my first try\PYZhy{}except }
\PY{n}{i} \PY{o}{=} \PY{l+m+mi}{5}; \PY{n}{j} \PY{o}{=} \PY{l+m+mf}{5.0}
        
\PY{k}{try}\PY{p}{:}
    \PY{c+c1}{\PYZsh{} the \PYZsq{}try\PYZsq{} block starts here}
    \PY{n}{k} \PY{o}{=} \PY{l+s+s1}{\PYZsq{}}\PY{l+s+s1}{5}\PY{l+s+s1}{\PYZsq{}}
    \PY{k}{print}\PY{p}{(}\PY{n}{i} \PY{o}{+} \PY{n}{j}\PY{p}{)}
    \PY{k}{print}\PY{p}{(}\PY{n}{i} \PY{o}{+} \PY{n}{k}\PY{p}{)}
    \PY{k}{print}\PY{p}{(}\PY{l+s+s1}{\PYZsq{}}\PY{l+s+s1}{success}\PY{l+s+s1}{\PYZsq{}}\PY{p}{)}
\PY{k}{except}\PY{p}{:}
    \PY{c+c1}{\PYZsh{} this new block will capture any}
    \PY{c+c1}{\PYZsh{} exception from the block above}
    \PY{k}{print}\PY{p}{(}\PY{l+s+s1}{\PYZsq{}}\PY{l+s+s1}{...silence...}\PY{l+s+s1}{\PYZsq{}}\PY{p}{)}
        
\PY{c+c1}{\PYZsh{} but this is no longer in the try block...}
\PY{k}{print}\PY{p}{(}\PY{n}{i} \PY{o}{+} \PY{n}{k}\PY{p}{)}
\end{Verbatim}
\end{frame}
\begin{frame}[fragile]
\justifying
output: 
 \begin{Verbatim}[commandchars=\\\{\}]
10.0
{\ldots}silence{\ldots}

---------------------------------------------------------------------------

TypeError         Traceback (most recent call last)

<ipython-input-4-57c7a9612f99> in <module>()
14 
15 \# but this is no longer in the try block{\ldots}
---> 16 print(i + k)
    
TypeError: unsupported operand type(s) for +: 
'int' and 'str'
    \end{Verbatim}
\end{frame}

\subsubsection{Functions and flow
controls}\label{functions-and-flow-controls}

\begin{frame}[fragile]
\frametitle{Functions and flow controls} 
\justifying
\textbf{flow controls} are structure to introduce conditions on the
execution of the code. Python implements the most common statements is a
compact way. 
\\~\\
The most usefull flow statements are 
\begin{itemize}
\item \textbf{if...else}
\item a \textbf{for}
\item a \textbf{while}
\end{itemize}
\end{frame}

\begin{frame}[fragile]
\justifying
\frametitle{The if - else statement} 
    \begin{Verbatim}[commandchars=\\\{\}]
\PY{c+c1}{\PYZsh{} a simple if statement:}
\PY{n}{i} \PY{o}{=} \PY{n+nb+bp}{True}
         
\PY{c+c1}{\PYZsh{} notice the == (comparison operator)}
\PY{k}{if} \PY{n}{i} \PY{o}{==} \PY{n+nb+bp}{True}\PY{p}{:}                              
    \PY{k}{print}\PY{p}{(}\PY{l+s+s1}{\PYZsq{}}\PY{l+s+s1}{this is True}\PY{l+s+s1}{\PYZsq{}}\PY{p}{)}
\PY{k}{else}\PY{p}{:}
    \PY{k}{print}\PY{p}{(}\PY{l+s+s1}{\PYZsq{}}\PY{l+s+s1}{this is False}\PY{l+s+s1}{\PYZsq{}}\PY{p}{)}
\end{Verbatim}
output: 
    \begin{Verbatim}[commandchars=\\\{\}]
this is True

    \end{Verbatim}
\end{frame}

\begin{frame}[fragile]
\justifying
\frametitle{The for statement} 
    \begin{Verbatim}[commandchars=\\\{\}]
\PY{c+c1}{\PYZsh{} the classical for loop:}
\PY{n}{factorial} \PY{o}{=} \PY{l+m+mi}{1}
         
\PY{k}{for} \PY{n}{i} \PY{o+ow}{in} \PY{n+nb}{range}\PY{p}{(}\PY{l+m+mi}{1}\PY{p}{,} \PY{l+m+mi}{10}\PY{p}{)}\PY{p}{:}
    \PY{n}{factorial} \PY{o}{=} \PY{n}{factorial} \PY{o}{*} \PY{n}{i} 
         
\PY{c+c1}{\PYZsh{} this is 9!}
\PY{k}{print}\PY{p}{(}\PY{n}{factorial}\PY{p}{)}                                
\end{Verbatim}
output: 
    \begin{Verbatim}[commandchars=\\\{\}]
362880

    \end{Verbatim}
\end{frame}

\begin{frame}[fragile]
\justifying
\frametitle{The while statement} 
\begin{Verbatim}[commandchars=\\\{\}]
\PY{c+c1}{\PYZsh{} the while loop}
\PY{n}{condition} \PY{o}{=} \PY{n+nb+bp}{True}
\PY{n}{i} \PY{o}{=} \PY{l+m+mi}{0}
\PY{k}{while} \PY{p}{(}\PY{n}{condition}\PY{p}{)}\PY{p}{:}
    \PY{n}{i} \PY{o}{=} \PY{n}{i} \PY{o}{+} \PY{l+m+mi}{1}
             
    \PY{k}{if} \PY{p}{(}\PY{n}{i} \PY{o}{\PYZgt{}}\PY{o}{=} \PY{l+m+mi}{10}\PY{p}{)}\PY{p}{:}
        \PY{n}{condition}  \PY{o}{=} \PY{n+nb+bp}{False}
         
\PY{k}{print}\PY{p}{(}\PY{n}{i}\PY{p}{)}
\end{Verbatim}
output:
    \begin{Verbatim}[commandchars=\\\{\}]
10
    \end{Verbatim}
\end{frame}

\begin{frame}
\justifying
\textbf{Functions} are defined using the \textbf{def} keyword and
\textbf{whitespace indentation} as defined above. 
\\~\\
Following the Python approach, function arguments are not explicitely typed, so their
behavior can be fairly affected by the arguments passed. It is however
very common to control variable types or values inside the function
\end{frame}

\begin{frame}[fragile]
\justifying
    \begin{Verbatim}[commandchars=\\\{\}]
\PY{c+c1}{\PYZsh{} my first function}
\PY{k}{def} \PY{n+nf}{add}\PY{p}{(}\PY{n}{i}\PY{p}{,} \PY{n}{j}\PY{p}{)}\PY{p}{:}
    \PY{k}{if} \PY{n+nb}{type}\PY{p}{(}\PY{n}{i}\PY{p}{)} \PY{o}{!=} \PY{n+nb}{type}\PY{p}{(}\PY{n}{j}\PY{p}{)}\PY{p}{:}
        \PY{k}{return} \PY{l+s+s1}{\PYZsq{}}\PY{l+s+s1}{the operation cannot be done !}\PY{l+s+s1}{\PYZsq{}}
    \PY{k}{return} \PY{n}{i} \PY{o}{+} \PY{n}{j}

\PY{n}{res1} \PY{o}{=} \PY{n}{add}\PY{p}{(}\PY{l+m+mi}{2}\PY{p}{,} \PY{l+m+mi}{3}\PY{p}{)}    \PY{c+c1}{\PYZsh{} adds...}
\PY{k}{print}\PY{p}{(}\PY{n}{res1}\PY{p}{)}
         
\PY{n}{res2} \PY{o}{=} \PY{n}{add}\PY{p}{(}\PY{l+s+s1}{\PYZsq{}}\PY{l+s+s1}{a}\PY{l+s+s1}{\PYZsq{}}\PY{p}{,} \PY{l+s+s1}{\PYZsq{}}\PY{l+s+s1}{b}\PY{l+s+s1}{\PYZsq{}}\PY{p}{)}    \PY{c+c1}{\PYZsh{} ...or concatenates}
\PY{k}{print}\PY{p}{(}\PY{n}{res2}\PY{p}{)}
         
\PY{c+c1}{\PYZsh{} Wait... What should we do here ?}
\PY{n}{res3} \PY{o}{=} \PY{n}{add}\PY{p}{(}\PY{l+s+s1}{\PYZsq{}}\PY{l+s+s1}{1}\PY{l+s+s1}{\PYZsq{}}\PY{p}{,} \PY{l+m+mi}{2}\PY{p}{)}
\PY{k}{print}\PY{p}{(}\PY{n}{res3}\PY{p}{)}
\end{Verbatim}
\end{frame}

\begin{frame}[fragile]
\justifying
output:
     \begin{Verbatim}[commandchars=\\\{\}]
5
ab
the operation cannot be done !

    \end{Verbatim}
\end{frame}

\subsection{Data Structure}
\begin{frame}
\justifying
\frametitle{Data Structure} 
Python knows a number of compound data types, used to group values together in some logical way. 
\\~\\ 
Any data structure (or container) is designed to organize data to suit a specific purpose so that it can be accessed and worked with in appropriate ways.
\\~\\ 
We'll introduce 3 basic data structures
\begin{itemize}
\item List
\item Tuple
\item Dictionary
\end{itemize}
\end{frame}

\subsubsection{List}
\begin{frame}
\justifying
\frametitle{List} 
The most versatile data container is the \textbf{list},
which can be written as a list of comma-separated values (items) between
\textbf{square brackets}: 
\begin{itemize}
\item list elements are indexed by a 0-based integer,
thus they can be \textbf{accessed randomly} (we can access immediately any
 element)
\item Lists might contain items of different types, but usually the items all have the same type.
\item Python comes with some usefull default list function, such as \textbf{range}. It also features 
\textbf{list comprehesions} to dramatically simplify list operations.
\end{itemize}
\end{frame}

\begin{frame}
\justifying
\frametitle{List} 
The most versatile data container is the \textbf{list},
which can be written as a list of comma-separated values (items) between
\textbf{square brackets}: 
\begin{itemize}
\item list elements are indexed by a 0-based integer,
thus they can be \textbf{accessed randomly} (we can access immediately any
 element)
\item Lists might contain items of different types, but usually the items all have the same type.
\item Python comes with some usefull default list function, such as \textbf{range}. It also features 
\textbf{list comprehesions} to dramatically simplify list operations.
\end{itemize}
\end{frame}

\begin{frame}[fragile]
\justifying
    \begin{Verbatim}[commandchars=\\\{\}]
\PY{c+c1}{\PYZsh{} a basic list declaration}
\PY{n}{l} \PY{o}{=} \PY{p}{[}\PY{l+m+mi}{10}\PY{p}{,} \PY{l+m+mi}{20}\PY{p}{,} \PY{l+m+mi}{5}\PY{p}{,} \PY{o}{\PYZhy{}}\PY{l+m+mi}{5}\PY{p}{,} \PY{l+m+mi}{10}\PY{p}{]}
\PY{k}{print}\PY{p}{(}\PY{n}{l}\PY{p}{)}
         
\PY{c+c1}{\PYZsh{} accessing the elements of the list is easy}
\PY{k}{print}\PY{p}{(}\PY{n}{l}\PY{p}{[}\PY{l+m+mi}{4}\PY{p}{]}\PY{p}{)}
         
\PY{c+c1}{\PYZsh{} range, notice the exclusion of the last element !}
\PY{n}{r} \PY{o}{=} \PY{n+nb}{range}\PY{p}{(}\PY{l+m+mi}{0}\PY{p}{,} \PY{l+m+mi}{10}\PY{p}{,} \PY{l+m+mi}{1}\PY{p}{)}
\PY{k}{print}\PY{p}{(}\PY{n}{r}\PY{p}{)}
         
\PY{c+c1}{\PYZsh{} how list comprehension works}
\PY{n}{square} \PY{o}{=} \PY{p}{[}\PY{n}{x} \PY{o}{*} \PY{n}{x} \PY{k}{for} \PY{n}{x} \PY{o+ow}{in} \PY{n}{r}\PY{p}{]}
\PY{k}{print}\PY{p}{(}\PY{n}{square}\PY{p}{)}
\end{Verbatim}
\end{frame}

\begin{frame}[fragile]
\justifying
output:
    \begin{Verbatim}[commandchars=\\\{\}]
[10, 20, 5, -5, 10]
10
[0, 1, 2, 3, 4, 5, 6, 7, 8, 9]
[0, 1, 4, 9, 16, 25, 36, 49, 64, 81]
    \end{Verbatim}
\end{frame}

\subsubsection{Tuple}
\begin{frame}
\justifying
\frametitle{Tuple} 
Tuple is another traditional container. It represent a group of
logically dependant elements. The difference with the list is that there
are \textbf{immutable}, e.g. once allocated, they cannot be modified nor
appended. 
\\~\\
Operations on Tuple can be performed by mapping there elements
using the \textbf{zip} statement
\end{frame}

\begin{frame}[fragile]
\justifying
\begin{Verbatim}[commandchars=\\\{\}]
\PY{n}{xyz} \PY{o}{=} \PY{p}{(}\PY{l+m+mf}{2.0}\PY{p}{,} \PY{l+m+mf}{1.7}\PY{p}{,} \PY{o}{\PYZhy{}}\PY{l+m+mf}{2.0}\PY{p}{)}
\PY{n}{abc} \PY{o}{=} \PY{p}{(}\PY{l+m+mf}{1.4}\PY{p}{,} \PY{l+m+mf}{3.6}\PY{p}{,} \PY{l+m+mf}{2.2}\PY{p}{)}
         
\PY{c+c1}{\PYZsh{} we can retrieve elements}
\PY{k}{print}\PY{p}{(}\PY{l+s+s2}{\PYZdq{}}\PY{l+s+s2}{x: }\PY{l+s+s2}{\PYZdq{}} \PY{o}{+} \PY{n+nb}{str}\PY{p}{(}\PY{n}{xyz}\PY{p}{[}\PY{l+m+mi}{0}\PY{p}{]}\PY{p}{)}\PY{p}{)}
         
\PY{c+c1}{\PYZsh{} simple coordinate manipulation}
\PY{n}{center} \PY{o}{=} \PY{p}{[}\PY{n+nb}{sum}\PY{p}{(}\PY{n}{x}\PY{p}{)} \PY{o}{/} \PY{l+m+mi}{2} \PY{k}{for} \PY{n}{x} \PY{o+ow}{in} \PY{n+nb}{zip}\PY{p}{(}\PY{n}{xyz}\PY{p}{,} \PY{n}{abc}\PY{p}{)}\PY{p}{]}
\PY{k}{print}\PY{p}{(}\PY{n}{center}\PY{p}{)}
         
\PY{c+c1}{\PYZsh{} violates immutability !}
\PY{n}{xyz}\PY{p}{[}\PY{l+m+mi}{0}\PY{p}{]} \PY{o}{=} \PY{o}{\PYZhy{}}\PY{l+m+mf}{1.0}
\end{Verbatim}
\end{frame}

\begin{frame}[fragile]
\justifying
output:
    \begin{Verbatim}[commandchars=\\\{\}]
---------------------------------------------

TypeError           Traceback (most recent call last)

<ipython-input-32-c6e69ed4e115> in <module>()
10 
11 \# violates immutability !
---> 12 xyz[0] = -1.0
    

TypeError: 'tuple' object does not support item 
assignment
    \end{Verbatim}
\end{frame}

\subsubsection{dictionary}
\begin{frame}
\justifying
\frametitle{Dictionary} 
In many cases, list are unsuitable to retrieve elements (from a logical
or performance point-of-view). 
This last container maps a list of 'keys' to another list of 'values'.
\\~\\ Dictionaries keeps the elements
\textbf{sorted} at all time. However, the keys have to be \textbf{unique}. They
can be strings, number or \textbf{Tuples}.
\end{frame}

\begin{frame}[fragile]
\justifying
\begin{Verbatim}[commandchars=\\\{\}]
\PY{n}{age} \PY{o}{=} \PY{p}{\PYZob{}}
\PY{p}{(}\PY{l+s+s1}{\PYZsq{}}\PY{l+s+s1}{John}\PY{l+s+s1}{\PYZsq{}}\PY{p}{,} \PY{l+s+s1}{\PYZsq{}}\PY{l+s+s1}{Doe}\PY{l+s+s1}{\PYZsq{}}\PY{p}{)}    \PY{p}{:} \PY{l+m+mi}{25}\PY{p}{,}
\PY{p}{(}\PY{l+s+s1}{\PYZsq{}}\PY{l+s+s1}{Jane}\PY{l+s+s1}{\PYZsq{}}\PY{p}{,} \PY{l+s+s1}{\PYZsq{}}\PY{l+s+s1}{Smith}\PY{l+s+s1}{\PYZsq{}}\PY{p}{)}  \PY{p}{:} \PY{l+m+mi}{27}\PY{p}{,}
\PY{p}{(}\PY{l+s+s1}{\PYZsq{}}\PY{l+s+s1}{Jack}\PY{l+s+s1}{\PYZsq{}}\PY{p}{,} \PY{l+s+s1}{\PYZsq{}}\PY{l+s+s1}{Morgan}\PY{l+s+s1}{\PYZsq{}}\PY{p}{)} \PY{p}{:} \PY{l+m+mi}{25}\PY{p}{,}
\PY{p}{(}\PY{l+s+s1}{\PYZsq{}}\PY{l+s+s1}{John}\PY{l+s+s1}{\PYZsq{}}\PY{p}{,} \PY{l+s+s1}{\PYZsq{}}\PY{l+s+s1}{Doe}\PY{l+s+s1}{\PYZsq{}}\PY{p}{)}    \PY{p}{:} \PY{l+m+mi}{24}       \PY{c+c1}{\PYZsh{} bad}
\PY{p}{\PYZcb{}}
        
\PY{k}{print}\PY{p}{(}\PY{n}{age}\PY{p}{)}
        
\PY{n}{key} \PY{o}{=} \PY{p}{(}\PY{l+s+s1}{\PYZsq{}}\PY{l+s+s1}{Jack}\PY{l+s+s1}{\PYZsq{}}\PY{p}{,} \PY{l+s+s1}{\PYZsq{}}\PY{l+s+s1}{Morgan}\PY{l+s+s1}{\PYZsq{}}\PY{p}{)}
\PY{k}{print}\PY{p}{(}\PY{l+s+s2}{\PYZdq{}}\PY{l+s+se}{\PYZbs{}n}\PY{l+s+s2}{ \PYZob{}0\PYZcb{} \PYZob{}1\PYZcb{} is \PYZob{}2\PYZcb{} years}\PY{l+s+s2}{\PYZdq{}}\PY{o}{.}\PY{n}{format}\PY{p}{(}
	\PY{n}{key}\PY{p}{[}\PY{l+m+mi}{0}\PY{p}{]}\PY{p}{,} \PY{n}{key}\PY{p}{[}\PY{l+m+mi}{1}\PY{p}{]}\PY{p}{,} \PY{n}{age}\PY{p}{[}\PY{n}{key}\PY{p}{]}\PY{p}{)}\PY{p}{)}
\end{Verbatim}
\end{frame}

\begin{frame}[fragile]
\justifying
output:
\begin{Verbatim}[commandchars=\\\{\}]
\{  
  ('Jack', 'Morgan'): 25
, ('John', 'Doe'): 24
, ('Jane', 'Smith'): 27
\}

 Jack Morgan is 25 years
\end{Verbatim}
\end{frame}
\end{document}
